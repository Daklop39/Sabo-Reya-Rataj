\documentclass{article}
\usepackage{graphicx} % Required for inserting images

\title{Automatizovaný dohľadový systém pre identifikáciu podvodov pri testovaní študentov ASFID}
\author{Michal Rataj, Daniel Reya, Samuel Sabo}
\date{September 2025}

\begin{document}

\maketitle
\section{Anotácia}
Projekt sa zameriava na výskum, návrh a implementáciu automatizovaného dohľadováho systému, ktorého cieľom je identifikovať a zameziť nepovolenému používaniu externých pomôcok počas online testovania. V súčasnosti sa vzdelávacie inštitúcie snažia o digitalizáciu, čo prináša novú problematiku v oblasti zabezpečenia férových podmienok pre všetkých študentov. Tradičné formy dozoru na nepovolené používanie externých pomôcok je pri digitalizovaní testovacích procesov často neefektívne, preto je potrebné hladať moderné automatizované riešenia, pomocou ktorých sa viedia zaistiť férové podmienky pre všetkých.\hfill \break

Navrhnutý systém bude kombinovať viaceré technológie vrátane analýzy obrazu, zvuku a okolia a následného vyhodnocovania získaných dát v reálnom čase. Pomocou globálnych kamier, web kamery a nahrávania obrazovky počítača v reálnom čase , systém bude schopný presne a rýchlo sledovať anomálie v správaní používateľa, sledovanie pohybu očí, analýzu vzorov klávesnice a myši, ako aj hodnotenie konzistencie odpovedí počas testu. Cieľom je vytvoriť viacvrstvový model, ktorý dokáže v reálnom čase identifikovať akékoľvek podozrivé aktivity. Systém je navrhnutý tak aby plne rešpektoval súkromie a komfort sledovaných ľudí a aby spĺňal etické zásady používania umelej inteligencie a zásady monitorovania osôb.\hfill \break


Projekt sa bude venovať legislatívnym ale aj metodickým aspektom využitia systému. Výsledkom bude prototyp softvérovo-hardverového riešenia, ktoré uľahčí takýmto inštitúciam efektívnejšie monitorovanie priebehu testovania. \hfill \break


Cieľom projektu je prispieť k rozvoju digitalizovanie v akademických inštitúciach na Slovensku ale aj v Európskych krajinách, s možnosťou následnej integrácie do existujúcich akademických systémov jednotlivých inštitoúcií.\hfill \break

\end{document}